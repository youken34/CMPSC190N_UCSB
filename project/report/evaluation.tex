\section{Evaluation}
\label{sec:eval}
Below is a good checklist for evaluation.
\begin{itemize}
    \item Appropriate figures of merit to evaluate the work are identified and motivated.
    \item Figures of merit are measured given a comprehensive range of practical parameters / operating conditions.
    \item Experimental setup is described sufficiently for a reader to replicate the testbed.
    \item Conclusions about the core insight of the paper make sense and draw cleanly from the experimental data.
    \item Design decisions are evaluated independently; role of each design choice is backed up with experimental data.
\end{itemize}

In the preamble of your evaluation section, provide a summary of questions that you aim to answer and preview of key results. The first sub-section of evaluation typically describes the setup, which includes description of dataset(s), testbed(s), tools, etc. I highly recommend WALTERing your graphs. 
\begin{itemize}
    \item \textbf{W}hy: why are we looking at this graph
    \item \textbf{A}xes: clearly describe the two axes and measurement units for each
    \item \textbf{L}ines: clearly describe what different lines/shapes represent in the graph
    \item \textbf{T}rends: What clear trends are visible in the graph, what's the implication
    \item \textbf{E}xceptions: clarify if there are some exceptions to the expected trend, explain why
    \item \textbf{R}ecap: provide a takeaway message and try to connect this with the motivation (part of Why).  Are the results as expected? What did we learn from it this experiment?
\end{itemize}

As a rule of thumb, there should only be one main takeaway from a graph. 





\begin{table}[t]
\begin{footnotesize}
\begin{center}
\resizebox{\linewidth}{!}{%
\begin{tabular}{|l| c c  c|}
\hline
\textbf{} & 
\textbf{\begin{tabular}[c]{@{}c@{}}Is\\ Realizable\end{tabular}} &
\textbf{\begin{tabular}[c]{@{}c@{}}Query\\ Planning\end{tabular}} &
\textbf{\begin{tabular}[c]{@{}c@{}}Stream Processor\\ Load\end{tabular}} 
\\

Static-MW  & \cmark & Static & 39.7~M \\ 
\hline
\system-Oracle & \xmark & Dynamic & 37.1~K \\
\system-Pred & \cmark & Dynamic & \textbf{39.6~K} \\
\hline
Optimal-Sonata & \xmark & Dynamic & 21.9~K 
\\ 
\hline
\end{tabular}
}
\end{center}
\end{footnotesize}
% \caption{Query-planning techniques emulated for evaluation.}
\caption{Sample table~\cite{sonata}.
% \ag{TODO: remove refinement column, update the load column.}
}
\label{tab:query_plans} 
\vspace{-.35in}
\end{table}



\begin{figure}[t] 
\begin{minipage}{1\linewidth}
\includegraphics[width=.9\linewidth]{1a.pdf}
\end{minipage}
\caption{Sample figure (results).
\label{fig:cost_of_not_adapting}
}
\end{figure}