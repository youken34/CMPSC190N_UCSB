\begin{abstract}
\textit{A good abstract should have the following key components: }
\begin{itemize}
    \item \textbf{Common ground}: \textit{Here, you should specify beliefs, assumptions, etc. that are acceptable to readers as unproblematic. For example, results from prior research that is generally accepted in the community. Any story, anecdote, or incident that is relevant to the paper but represents stability. The purpose of this text is to contextualize what follows, but most importantly, it sets up a situation just so that it can be disrupted. }\hl{Traffic classification has been widely studied, with many models achieving significant success using features like packet size, frequency, and flow duration. Datasets such as ISCX VPN-Tor and CICIDS are commonly used and are accepted in the community as benchmarks for model evaluation. The general practice of using packet-level data for traffic classification is well-established and considered reliable.}
    \item \textbf{Disruption}: \textit{Here, you present the destabilizing condition. The destabilizing condition can be errors in prior works/understanding, false assumptions, ignorance, contradictions, etc.}\hl{However, these established models and datasets often fail to address domain-specific classifications, such as distinguishing between gaming and music traffic, which could have unique network signatures. Additionally, most public datasets lack labels for such granular distinctions, creating a gap in real-world applicability.}
    \item \textbf{Cost}: Here, you highlight the cost of destabilizing conditions. What do we lose by sticking with the status quo? \hl{Without addressing this gap, network providers and users miss out on optimized bandwidth allocation for highly specific applications, potentially leading to poorer user experiences and higher operational costs.}
    \item \textbf{Resolution}: Here, you talk about your approach to solving the problem described above. You provide a gist of your solution, and an intuition on why you think it will work. The intuition can be based on new observations or an overview of results. 
\end{itemize}



\end{abstract}
