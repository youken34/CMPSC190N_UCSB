\begin{abstract}
\textit{A good abstract should have the following key components: }
\begin{itemize}
    \item \textbf{Common ground}: \textit{Here, you should specify beliefs, assumptions, etc. that are acceptable to readers as unproblematic. For example, results from prior research that is generally accepted in the community. Any story, anecdote, or incident that is relevant to the paper but represents stability. The purpose of this text is to contextualize what follows, but most importantly, it sets up a situation just so that it can be disrupted. }\hl{Music videos obviously require less bandwidth to load than the gaming ones, furthermore, video games graphics and image quality have significantly increased over the years and are expected to continue advancing, resulting in distinct networking demands.}
    \item \textbf{Disruption}: \textit{Here, you present the destabilizing condition. The destabilizing condition can be errors in prior works/understanding, false assumptions, ignorance, contradictions, etc.}\hl{The first struggle we went through came along with the range of data we were working on, since the provided dataset is limited to mere "videos", we were unable to accurately pick two videos respectively labeled as "gaming" and "audio only". Secondly, we ran into a server issue which prevented us from connecting to the right aws-fargate. Finally, we faced problems with files not updating correctly, as well as permission issues when attempting to modify.}
    \item \textbf{Cost}: \textit{Here, you highlight the cost of destabilizing conditions. What do we lose by sticking with the status quo?} \hl{The consequences of the aforementioned roadblocks we encountered result  in inefficient resource allocation, as resources are wasted on videos that do not require as much resources to play.}
    \item \textbf{Resolution}: \textit{Here, you talk about your approach to solving the problem described above. You provide a gist of your solution, and an intuition on why you think it will work. The intuition can be based on new observations or an overview of results.} \hl{In order to find accurate data we could train our model on, we manually picked two video, one for each category our machine learning model aims to classify. Regarding the server issue, we solved it by trying differents servers / node. Lastly, when it comes to the files problems, we managed the find the correct paths to get rid of this issue.}
\end{itemize}



\end{abstract}
