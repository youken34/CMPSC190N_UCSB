\section{Introduction}
\label{sec:intro}
 

\textbf{Paragraph 1}: Video games and streaming have experienced fancier graphics and resolution over the last decades, therefore, it is relevant to ensure if the network traffic classification still works efficiently as it used to. The performed classifications could result as a better allocation of resources allowing us to manage the bandwidth in the most optimal way to avoid wasting it. 


\textbf{Paragraph 2}: Nowadays, the fast growth of online activities such as streaming and gaming puts immense demands on the network infrastructure. Traditional approaches to traffic classification cannot stand up to the complexity and diversity of data flows. Our project is aimed at addressing this challenge by leveraging machine learning to differentiate between gaming and music video traffic coming from the same source (youtube) in order to make networks efficient enough to meet the needs of our time.


\textbf{Paragraph 3}: In this paper, we came up with a new approach to traffic classification, which differentiates between gaming and music video, both sourced from YouTube. By using our own dataset and machine learning techniques, our work introduces a classification method by identifying these distinct traffic types. First, instead of training our model on mock data, we consider the real-world traffic data and, through a pipeline performing the key features one by one, such as packet size / frequency, data spikes. Our findings reveal that differentiating between gaming and music videos is not just about optimizing network resource allocation, it represents an significant step made toward redesigning and modernizing networks to align with the current era we live in. By addressing this gap, our project lays the groundwork for  smarter and more flexible networking solutions in accordance to the digital era.


\textbf{Paragraph 4}: We came across a related study called Beauty and the Burst, which explores the burstiness of network traffic and how it affects traffic characterization and quality of service. The study offers valuable insights into how bursty behavior impacts network performance across a wide range of applications. However, our approach is different. While Beauty and the Burst focuses on the pattern of traffic at a larger scale, our contribution focuses on the classification of gaming and music video traffic over platforms like YouTube. By using machine learning and a custom dataset specifically designed for this purpose, we restrict ourselves to diving deeper into the unique characteristics of these traffic types and exploring ways to optimize resource allocation more effectively. This focused approach highlights the unique contributions of our work.


\textbf{Paragraph 5}: The rest of this paper is organized as follows. The second section give further insights on the project's purpose and which needs it aims to satisfy explaining how relevant it is in the given context. In the third section we will be describing our approach toward the expected result, breaking it down into different stages such as the data collection process and how we plan to handle them. Throughout the section 4, we will dive deeper into the technical details, focusing on the implementation of our machine learning model specifically. The fifth section will be related to our findings, we will be discussing their relevancy as well as we will compare them to the results expected initially, this section will also be devoted to provide any pertinent graph which is related to our work. Lastly, in Section 6, we summarize the essential conclusions of this work, discuss potential challenges faced, and provide thoughtful suggestions for advancing this research in the future.






