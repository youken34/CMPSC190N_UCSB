\section{Background and Motivation}
\label{sec:background}





Network traffic classification is one of the important ways to improve the efficiency of network resources and optimize network performance. More fundamentally, this research field focuses on the analysis of data packets sent by varying activities such as video streaming, web browsing, or gaming taking place over a network. Traditional methods typically rely on simple features such as packet size, frequency, and flow duration to classify network traffic, however, the rapid advancement of modern technologies, particularly in the gaming field, has introduced unique challenges making these approaches less and less relevant over time.

\vspace{2mm}

The specific problem addressed in this paper lies in distinguishing between gaming and music video traffic, both sourced from the same platform (YouTube). Different patterns are observed through our guided experience. Despite their differences, traditional classification approaches often fail to capture these nuances, treating them as part of broader video traffic, whereas, our goal was to focus on the accuracy and relevancy of the results by performing tests and training machine learning models on smaller-scale project and well-defined sub-category.

Our approach is motivated by several unique observations:
\begin{itemize}
    \item Gaming and music video traffic exhibit unique traffic patterns, such as variations in packet frequency, latency, and data spikes, which provide us with insightful information that has yet to be utilized in our project enabling us to draw conclusions from
    \item There is an abundance of videos available on YouTube that we can use to train our model, making it possible for us to bring this project to life especially since this is a small-scale project.
    \item This, combined with the advanced machine learning technologies mainly used across various fields, including network traffic analysis, helps to bring about accurate and adaptive techniques in classification, which are precisely developed for the challenges in today's applications.
\end{itemize}

While existing works, such as *Beauty and the Burst*, focus on analyzing traffic patterns at a macro level, our approach narrows the scope to domain-specific classification, focusing uniquely on videos coming from one source which is youtube. Current solutions often fail to address the unique characteristics of gaming and music traffic, making them less effective for application-specific optimization.

\vspace{2mm}

By designing our own custom dataset, picking manually the videos our model is going to work on  , managing our own machine learning techniques, our project addresses these limitations directly. This customized method improves classification precision while paving the way for a reimagined approach to network resource allocation, offering better support for the evolving needs of today's applications. This work is an important step toward closing the gap between detailed traffic analysis and the particular application needs.